%%%%%%%%%%%%%%%%%%%%%%%%%%%%%%%%%%%%%%%%%
% Professional Formal Letter
% LaTeX Template
% Version 2.0 (12/2/17)
%
% This template originates from:
% http://www.LaTeXTemplates.com
%
% Authors:
% Brian Moses
% Vel (vel@LaTeXTemplates.com)
%
% License:
% CC BY-NC-SA 3.0 (http://creativecommons.org/licenses/by-nc-sa/3.0/)
%
%%%%%%%%%%%%%%%%%%%%%%%%%%%%%%%%%%%%%%%%%

%----------------------------------------------------------------------------------------
%	PACKAGES AND OTHER DOCUMENT CONFIGURATIONS
%----------------------------------------------------------------------------------------

\documentclass[12pt, a4paper]{simref} % Set the font size (10pt, 11pt and 12pt) and paper size (letterpaper, a4paper, etc)


%\longindentation=0pt % Un-commenting this line will push the closing "Sincerely," and date to the left of the page

%----------------------------------------------------------------------------------------
%	YOUR INFORMATION
%----------------------------------------------------------------------------------------

\Who{Prof. Dr. Ir. Atik Suprapti} % Your name

\Title{,M.T.} % Your title, leave blank for no title

\authordetails{
	Universitas Diponegoro\\ % Your department/institution
	Jl. Prof. Soedarto, Kampus Undip\\ % Your address
	Semarang\\ % Your city, zip code, country, etc
	Email: atiksuprapti@arsitektur.undip.ac.id\\ % Your email address
	Phone: 08156624006\\ % Your phone number
	%URL: LaTeXTemplates.com % Your URL
}

%----------------------------------------------------------------------------------------
%	HEADER CONTENTS
%----------------------------------------------------------------------------------------

\logo{logo1.png} % Logo filename, your logo should have square dimensions (i.e. roughly the same width and height), if it does not, you will need to adjust spacing within the HEADER STRUCTURE block in structure.tex (read the comments carefully!)

\headerlineone{} % Top header line, leave blank if you only want the bottom line

\headerlinetwo{Universitas Diponegoro} % Bottom header line

%----------------------------------------------------------------------------------------

\begin{document}

%----------------------------------------------------------------------------------------
%	TO ADDRESS
%----------------------------------------------------------------------------------------

\begin{letter}{
	Scholarship review committee\\
	Ministry of Education\\
	Taiwan\\
}

%----------------------------------------------------------------------------------------
%	LETTER CONTENT
%----------------------------------------------------------------------------------------

\opening{Dear Scholarship review committee,}

It is my pleasure to recommend M. Uliah Shafar for your scholarship program. Uliah was one of my student in master degree at Universitas Diponegoro. I had taught him few subjects of architecture and also supervised him in his thesis. While I was interacting with him, he had successfully showed courage, enthusiasm, and hard work to get his task done. In my point of view, this made him as an outstanding student. Moreover, he had proven that by graduated with highest GPA and cum laude graduate.

We have made some interaction in the class with other classmates in few subjects for nearly one semester. However, after Covid breakdown, the class should be conducted online. Although some students showed lack of interest in online class, Uliah increased his discipline, attendance and completed task punctually. He is very flexible and reliable person, no matter how different the situation are. In every class meeting, before or after Covid, Uliah always comes with groundbreaking idea, either by writing assignment, discussion or presentation. The class even more interesting, when he tried to involved by raising question, giving opinion, or commenting their classmate's ideas. He actively engaged with compassion and respect to all in his study group. His contribution to full-class discussion give clear insight both for his peers and for me as his lecturer.

In his writing task, he wrote comprehensive and accurate ideas to any other topic that was given.
He is such the best writer
even though he must catch the deadline of more than a single task weekly. The immersion in improving his writing made his one article published in national journal of urban design in early stage of education. The article that he wrote is about the public space evaluation.
This is not only show that he care only for his education but also the impact for society.

I am completely sure that Uliah would make a good impression in his future education.
If the accomplishment of his academic endeavor is a good indication of how he would be seen in days to come, he would be a positive asset to your scholarship. As a lecture who is willing to take a part of student's success, I fully encourage you to consider him for this scholarship.



%I have known Uliah since he attended some my master's subjects in Universitas Diponegoro. My relationship was getting stronger when I had became supervisor of him for his thesis. While interaction, Uliah demonstrate good quality such as consistently accomplished tasks revision with dedication and enthusiast. Uliah showed an excelent  assignment writing with support of incredible and ramp up citation. He also one of students who graduated with camlaude. Therefore, I would like to recommend Uliah as prospective scholarship awardee.

\closing{Sincerely,}

%----------------------------------------------------------------------------------------
%	OPTIONAL FOOTNOTE
%----------------------------------------------------------------------------------------

% Uncomment the 4 lines below to print a footnote with custom text
%\def\thefootnote{}
%\def\footnoterule{\hrule}
%\footnotetext{\hspace*{\fill}{\footnotesize\em Footnote text}}
%\def\thefootnote{\arabic{footnote}}

%----------------------------------------------------------------------------------------

\end{letter}

\end{document}
